\documentclass{article}
\usepackage{amsfonts}
\usepackage{amsmath}
\usepackage{mathtools}

\title{Assignment 7}
\author{Owen Ritzau}
\date{April 8th, 2023}

\begin{document}
\maketitle
\section* {5.1}
\begin{enumerate}
  \item[2)]
    $\lambda = 2$ is an eigenvalue of $\begin{bmatrix*}[r]
      7 & 3 \\
      3 & -1 \\
    \end{bmatrix*}$ if $A-2 I_2=0$ 
    
    $A-2 I_2=\begin{bmatrix*}[r]
      5 & 3\\
      3 & -3 \\
    \end{bmatrix*}$

    $det(A) = 5(-3) - (3\cdot3) \neq 0$, therefore 
    $\lambda$ is not an eigenvalue of $A$

  \item[4)]
    $\begin{bmatrix*}
      -1 + \sqrt{2} \\
      1 \\
    \end{bmatrix*}$ is an eigenvector of $A$ if 
    $\begin{bmatrix*}
      2 & 1 \\
      1 & 4 \\
    \end{bmatrix*}\begin{bmatrix*}
      -1 + \sqrt{2} \\
      1 \\
    \end{bmatrix*} = \lambda \begin{bmatrix*}
      -1 + \sqrt{2} \\
      1 \\
    \end{bmatrix*}$

    $\begin{bmatrix*}
      2 & 1 \\
      1 & 4 \\
    \end{bmatrix*}\begin{bmatrix*}
      -1 + \sqrt{2} \\
      1 \\
    \end{bmatrix*} = 
    \begin{bmatrix}
      2(-1 + \sqrt{2}) + 1 \\
      -1 + \sqrt{2} + 4 \\
    \end{bmatrix} \approx 4.4142 \cdot \begin{bmatrix*}
      -1 + \sqrt{2} \\
      1 \\
    \end{bmatrix*}$ 

    Therefore, $\begin{bmatrix*}
      -1 + \sqrt{2} \\
      1 \\
    \end{bmatrix*}$ is an eigenvector of A with an 
    eigenvalue of 4.4142

  \item[8)]
    $det\begin{bmatrix*}[r]
      -2 & 2 & 2 \\
      3 & -5 & 1 \\
      0 & 1 & -2 \\
    \end{bmatrix*} = 0$ therefore $\lambda = 3$ is an 
    eigenvalue of $A$. 
    
    $\begin{bmatrix*}[r]
      -2 & 2 & 2 & 0\\
      3 & -5 & 1 & 0\\
      0 & 1 & -2 & 0\\
    \end{bmatrix*} \rightarrow \begin{bmatrix*}[r]
      1 & 0 & -3 & 0\\
      0 & 1 & -2 & 0\\
      0 & 0 & 0 & 0\\
    \end{bmatrix*} \rightarrow
    x_3\begin{bmatrix*}[r]
      3 \\ 2 \\ 1 
    \end{bmatrix*}$

    Any vector of the form $x_3\begin{bmatrix*}[r]
      3 \\ 2 \\ 1 
    \end{bmatrix*}$ where $x_3 \neq 0$ is an eigenvector of
    $A$.

  \item[10)]
    $A - \lambda I_2 = \begin{bmatrix*}[r]
      6 & -9 \\
      4 & -6 \\
    \end{bmatrix*}$

    $\begin{bmatrix*}[r]
      6 & -9 & 0\\
      4 & -6 & 0\\
    \end{bmatrix*} \rightarrow \begin{bmatrix*}[r]
      1 & -\frac{3}{2} & 0\\
      0 & 0 & 0\\
    \end{bmatrix*}$

    The basis for the eigenspace of $A$ is 
    $\left\{\begin{bmatrix*}[r]
        \frac{3}{2} \\
        1\\
    \end{bmatrix*}\right\}$

  \item[12)]
    $A - I_2 = 
    \begin{bmatrix*}[r]
      6 & 4 \\
      -3 & -2 \\
    \end{bmatrix*}$ 

    $A - 5I_2 = 
    \begin{bmatrix*}[r]
      2 & 4 \\
      -3 & -6 \\
    \end{bmatrix*}$

    $\lambda = 1$, basis $= \left\{
      \begin{bmatrix*}[r]
        -\frac{2}{3} \\
        1 \\
      \end{bmatrix*}
    \right\}$, $\lambda = 5$, basis $=\left\{
      \begin{bmatrix*}[r]
        -2 \\ 
        1 \\
      \end{bmatrix*}
    \right\}$

  \item[16)]
    basis =
    $\left\{
      \begin{bmatrix}
        2 \\
        3 \\
      \end{bmatrix}
    \right\}$

  \item[18)]
    $A$ is in lower triangular form, therefore for $\lambda$ to 
    be an eigenvalue of $A$, it must satisfy
    $(4 - \lambda)(0 - \lambda)(-3 - \lambda) = 0$

    Therefore, $\lambda = 4, \lambda = -3,$ and 
    $\lambda = 0$ are 
    eigenvalues of $A$.
  
  \item[22)]
    \begin{enumerate}
      \item
        False, this is the definition of an eigenvector however
        the vector cannot be the zero vector
      \item
        False, a matrix can have the same value across its 
        diagonal multiple times, creating multiple linearly 
        independent vectors with identical eigenvalues.
      \item 
        True, a steady state vector $\vec x$ is 
        a vector such that $A \vec x= x$.
        Therefore, $\vec x$ is an eigenvector with 
        corresponding eigenvalue 1.
      \item 
        False, this is true only for triangular matrices. 
        Because row operations alter eigenvalues, not every matrix
        has its eigenvalues across its diagonal. 
      \item 
        True, the "certain matrix" is $A - \lambda I$
  \end{enumerate}
\end{enumerate}
\section*{5.2}
\begin{enumerate}
  \item[2)]
\end{enumerate}
\end{document}
