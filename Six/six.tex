\documentclass{article}
\usepackage{amsfonts}
\usepackage{amsmath}
\usepackage{mathtools}

\title{MA200 Assignment 6}
\author{Owen Ritzau}
\date{March 27th, 2023}

\begin{document}
\maketitle
  \section* {4.2}
  \begin{enumerate}
    \item[16)]
      $A=\begin{bmatrix*}[r]
      1&-1&0\\
      2&1&1\\
      0&5&-4\\
      0&0&1\\
      \end{bmatrix*}$
    
    \item[18)]
      Because $A$ has 4 entries in each column, Col $A$ is a 
      subspace of $\mathbb{R}^4$. $A$ has 4 entries in each row,
      meaning that a valid vector x such that $Ax$ is defined
      must have 4 entries, therefore Nul $A$ is a subspace
      of $\mathbb{R}^4$

    \item[20)]
      A has 1 entry in each column, therefore Col $A$ is a
      subspace of $\mathbb{R}$. $A$ has 5 entries per row, 
      meaning that Nul $A$ is a subspace of $\mathbb{R}^5$

    \item[22)]
      $\begin{bmatrix*}[r]-6\\2\\0\\1\end{bmatrix*}$ is in 
      Nul $A$, 
      $\begin{bmatrix*}0\\1\\\end{bmatrix*}$ is in Col $A$
    
    \item[24)]
      Col $A$ can be defined as $x_3\begin{bmatrix*}[r]
        1\\.5\\-1\end{bmatrix*}. w = -2\cdot\begin{bmatrix*}[r]
        1\\.5\\-1\end{bmatrix*}$, therefore $w$ is in Col $A$.

      $A\cdot w \neq \begin{bmatrix*}0\\0\\0\\\end{bmatrix*}$,
      therefore $w$ is not in Nul $A$.
    \item[26)] 
      a. True, a null space has the zero vector and 
      maintains the laws of scalar addition and multiplication,
      therefore it is a vector space

      b. False, the null space is in $\mathbb{R}^n$ but not 
      necessarily $\mathbb{R}^m$

      c. False, Col $A$ is $\left\{b : b=Ax\right\}$

      d. True, the kernel refers specifically to the zero 
      vector, and the set $\{T:T(x)=0\}=$ Nul $A$

      e. True, the range will maintain addition and multiplication
      properties and will contain the zero vector

      f. True

  \end{enumerate}

  \section* {4.2}
  \begin{enumerate}

    \item[4)]
      The set composed of the columns of
      $\begin{bmatrix*}[r]
      2&1&-7\\
      -2&-3&5\\
      1&2&4\\
      \end{bmatrix*}$ is a basis for $\mathbb{R}^3$ as it is
      both linearly independent and spans $\mathbb{R}^3$

    \item[6)]
      The set composed of the columns of
      $\begin{bmatrix*}[r]
      1&-4\\
      2&-5\\
      -3&6\\
      \end{bmatrix*}$ is not a basis for $\mathbb{R}^3$ as it 
      does not span $\mathbb{R}^3$, however it is linearly independent 
    
    \item[8)]
      Because the matrix formed by the vectors of this set has
      more columns than rows, it is guaranteed that it will not
      be linearly independent and therefore is not a basis for
      $\mathbb{R}^3$. This set spans $\mathbb{R}^3$.

    \item[10)]

      $\begin{bmatrix*}[r]
      1&0&-5&1&4&0\\
      -2&1&6&-2&-2&0\\
      0&2&-8&1&9&0\\
      \end{bmatrix*}\rightarrow
      \begin{bmatrix*}[r]
      1&0&-5&0&7&0\\
      0&1&-4&0&6&0\\
      0&0&0&1&-3&0\\
      \end{bmatrix*}$

      $\begin{bmatrix*}[r]
      x_1-5x_3+7x_5=0\\
      x_2-4x_3+6x_5=0\\
      x_4-3x_5=0\\
      \end{bmatrix*}\rightarrow
      $ Nul $A=\left\{\begin{bmatrix*}[r]
      5\\4\\1\\0\\0\end{bmatrix*},
      \begin{bmatrix*}[r]
      -7\\-6\\0\\3\\1\end{bmatrix*}\right\}$
    
    \item[14)]
      Nul $A = \left\{\begin{bmatrix*}[r]
      -2\\1\\0\\0\\0\end{bmatrix*},
      \begin{bmatrix*}[r]
      -4\\0\\\frac{7}{5}\\1\\0\end{bmatrix*}
      \begin{bmatrix*}[r]
      -5\\-\frac{8}{5}\\0\\0\\1\end{bmatrix*}
      \right\},
      $ Col $A=\left\{
      \begin{bmatrix*}[r]1\\2\\1\\3\end{bmatrix*},
      \begin{bmatrix*}[r]-5\\-5\\0\\-5\end{bmatrix*}
      \right\}$

    \item[16)]
      The basis for the space spanned by vectors $v_1..v_5=
      \mathbb{R}^3$

    \item[22)]
      a. False, the set must span H to be a basis for H

      b. True, if S spans V then it is guaranteed to have some
      linearly independent vectors such that it is a basis for
      V.
  
      c. True, a basis cannot be any larger while maintaining
      linear independence

      d. False, the method will always produce a set

      e. False, the pivot columns are not always in the 
      column space of A

      \end{enumerate}


\end{document}
